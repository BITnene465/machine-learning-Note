% sections/section1.tex

\section{线性回归算法}


\subsection{简介}


\subsection{定义和数学方法}

\textbf{对于两个自变量的线性回归:}
\[Y = X_1 \theta_1 + X_2 \theta_2\]

\textbf{拟合的平面:}
\[h_\theta (x_1, x_2) = \theta_0 + \theta_1 x_1 + \theta x_2 \]

\textbf{一般性整合:}
\[h_\theta (\mathbf{x}) = \sum_{i = 1}^{n} \theta_i x_i =  \mathbf{\theta}^T \mathbf{x}\]

\textbf{误差(概率论基础):} 
\begin{itemize}
    \item 真实值和预测值之间的差异,一般记作 \(\varepsilon \)
    \item 对于每个样本,预测和误差: \(y^{(i)} = \theta^T x^{(i)} + \varepsilon^{(i)}\)
    \item 误差服从高斯分布:
    \[p(\varepsilon^{(i)}) = \frac{1}{\sqrt[]{2 \pi}\sigma} e^{-\frac{(\varepsilon^{(i)})^2}{2 \sigma^2}} \]
    \item 似然函数:
    \[ L(\theta) =  \prod_{i=1}^{m} p(y^{(i)}| x^{(i)};\theta)  \]
    \item 对数似然函数:
    \[ \log L(\theta)\]   
\end{itemize}

\textbf{对于对数似然函数展开化简:}

$$\sum_{i=1}^{m}\log\frac{1}{\sqrt[1]{2\pi}}\exp(-\frac{(y^{(i)}-\theta^Tx^{(i)})^2}{2\sigma^2})= m \log\frac{1}{\sqrt[]{2\pi}\sigma } - \frac{1}{\sigma^2}\cdot \frac{1}{2}\sum_{i=1}^{m}(y^{(i)}-\theta^T x^{(i)})^2 $$

\textbf{目标函数}
\begin{align*}
    J\big(\theta\big) & ={\frac{1}{2}}\sum_{i=1}^{m}(y^{(i)}-\theta^{T}x^{(i)})^{2}\\
    & ={\frac{1}{2}}\sum_{i=1}^{m}{\Big(}h_{\theta}{\big(}x^{(i)}{\big)}-y^{(i)}{\Big)}^{2}={\frac{1}{2}}{\big(}X\theta-y{\big)}^{T}{\big(}X\theta-y{\big)}\\
\end{align*} 

\textbf{求相应的参数}
$$\theta_{val} = argmin J(\theta)$$
\begin{align*}
    \nabla_{\theta}J(\theta) &= \nabla_{\theta}\Bigl({\frac{1}{2}}(X\theta-y)^{T}(X\theta-y)\Bigr) \\
    &= \nabla_{\theta}\Bigl({\frac{1}{2}}\Bigl(\theta^{T}X^{T}-y^{T}\Bigr)(X\theta-y)\Bigr) \\
    &= {\frac{1}{2}}\nabla_{\theta}\Bigl(\theta^{T}X^{T}X\theta-\theta^{T}X^{T}y-y^{T}X\theta+y^{T}y\Bigr) \\
    &= {\frac{1}{2}}\Bigl(2X^{T}X\theta-X^{T}y-\bigl(y^{T}X\bigr)^{T}\Bigr) \\
    &= X^{T}X\theta-y^{T}y
\end{align*}
令导数等于0, \qquad 则有 $\theta = (X^{T}X)^{-1} X^T y$

\subsection{优化算法--梯度下降法}

\textbf{引入:}得到一个目标函数后,如何进行求解?

直接求解?(并不一定可解,线性回归是一个有公式的特例)

\textbf{如何优化参数?}



\textbf{目标函数:}
$$J\big(\theta\big) = {\frac{1}{2m}}\sum_{i=1}^{m}{\Big(}h_{\theta}{\big(}x^{(i)}{\big)}-y^{(i)}{\Big)}^{2}$$

\textbf{学习率(LR):}对结果会产生巨大的影响,一般需要小一些

\textbf{批处理数量(batch):}32,64,128都可以

\begin{itemize}
    \item 批量梯度下降(GD):$$\frac{\partial J(\theta)}{\partial \theta_{j}}=-\frac{1}{m}\sum_{i=1}^{m}(y^{i}-h_{\theta}(x^{i}))x_{j}^{i}\qquad \theta_{j}^{\;'}=\theta_{j}+\frac{1}{m}\sum_{i=1}^{m}(y^{i}-h_{\theta}(x^{i}))x_{j}^{i}$$  \newline 容易得到最优解,但是速度慢
    \item 随机梯度下降(SGD):$${\theta}_{j}^{'}=\theta_{j}+(y^{i}-h_{\theta}(x^{i}))x^{i}_{j}$$ \newline 迭代速度快,但是不一定每次都朝着收敛的方向
    \item 小批量梯度下降:$$\theta_{j}:=\theta_{j}-\alpha{\frac{1}{10}}\sum_{k=i}^{i+9}(h_{\theta}(x^{(k)})-y^{(k)})x_{j}^{(k)}$$ \newline  每次更新选择一小部分数据来算,比较实用!
\end{itemize}








% 添加更多的子节和内容...

\subsection{总结}



